
%%% Eingebundene Pakete: werden hier nicht alle gebraucht, schaden aber auch nicht.

\usepackage{enumitem}
\usepackage{array}
\usepackage{caption}

\usepackage{tabularx}

\usepackage{mdframed}

% \usepackage{todonotes}


\usepackage{glossaries} %fleqn %bibotoc
\usepackage{mathptmx}
% \usepackage[scaled=0.9]{helvet}
\usepackage[scaled]{helvet}
\usepackage{courier}

%\usepackage{lmodern}
%\usepackage{mathptmx}
\DeclareMathAlphabet{\mathcal}{OMS}{cmsy}{m}{n}

\usepackage{arydshln}
\usepackage{graphicx}
\usepackage[english]{babel}
\usepackage{extarrows}
\usepackage{amsmath,amsthm,amssymb,amsfonts,amscd,amsbsy,amsxtra,wasysym}
\usepackage{epsfig,color}
\usepackage{subfigure}
\usepackage{fancyhdr}
\usepackage{psfrag}
\usepackage{enumerate}
\usepackage[T1]{fontenc}
\usepackage[utf8]{inputenc}

\usepackage{cancel}
\usepackage{stmaryrd}
\usepackage{wrapfig}
\usepackage{caption}
\usepackage{blindtext}
\usepackage{hyperref}
\usepackage{url}
\usepackage{float}
\usepackage{mathtools}
\usepackage{tikz}
% \usetikzlibrary{external}
% \tikzexternalize[prefix=tikz/]
\usepackage{pgfplots}
\pgfplotsset{compat=1.18} %todo
\usepackage{booktabs}
\usepackage{multirow}

\usepackage{pdfpages}

\usepackage[Algorithmus]{algorithm}
\usepackage{algpseudocode}

%\usepackage{etex}
\usetikzlibrary{external} %pdflatex -shell-escape <filename> beim Kompilieren
\tikzexternalize[prefix=tikz/]


\usepackage[babel,german=quotes]{csquotes}
\usepackage[style=numeric-comp,giveninits=true,backend=biber]{biblatex}
%\usepackage[style=numeric-comp,giveninits=true,backend=bibtex8]{biblatex}
\addbibresource{ref.bib}


%%% Neudefinition von Kommandos, um sich Tipp-Arbeit zu ersparen
\newcommand{\Q}{\mathbb{Q}}
\newcommand{\R}{\mathbb{R}}
\newcommand{\N}{\mathbb{N}}
\newcommand{\Ra}{\Longrightarrow}
\newcommand{\La}{\Longleftarrow}
\newcommand{\LRa}{\Leftrightarrow}
\newcommand{\lr}[1]{\left(#1\right)}

\DeclareMathOperator*{\argmin}{arg\,min}
\DeclareMathOperator*{\argmax}{arg\,max}
\DeclareMathOperator*{\spn}{span}


%%% Veraenderung des Seitenlayouts
%\setlength{\oddsidemargin}{-1cm}
%\setlength{\evensidemargin}{0cm}
%\setlength{\textwidth}{18cm}
%\setlength{\textheight}{26cm}
%\setlength{\topmargin}{-2cm}

\setlength{\parindent}{0pt}   % kein linker Einzug der ersten Absatzzeile
%\setlength{\parskip}{1.4ex plus 0.35ex minus 0.3ex} % Absatzabstand, leicht variabel
\pagestyle{plain}

%\usepackage[paper=a4paper,left=20mm,right=20mm,top=20mm,bottom=20mm]{geometry}
\usepackage[paper=a4paper,left=35mm,right=25mm,top=30mm,bottom=30mm]{geometry}
%\usepackage[onehalfspacing]{setspace}

%%%% von Uni Vorlage
\usepackage{setspace}         % Paket für div. Abstände, z.B. ZA
%\onehalfspacing              % nur dann, wenn gefordert; ist sehr groß!!
\setlength{\parindent}{0pt}   % kein linker Einzug der ersten Absatzzeile
\setlength{\parskip}{1.4ex plus 0.35ex minus 0.3ex} % Absatzabstand, leicht variabel

%\sloppy
\allowdisplaybreaks

\theoremstyle{definition}
\newtheorem{bsp}{Beispiel}[section]
\newtheorem{defi}[bsp]{Definition}
\newtheorem{bem}[bsp]{Bemerkung}
\newtheorem{folg}[bsp]{Folgerung}
\newtheorem{an}[bsp]{Annahme}

\theoremstyle{plain}
\newtheorem{satz}[bsp]{Satz}
\newtheorem{theorem}[bsp]{Theorem}
\newtheorem{lemma}[bsp]{Lemma}
\newtheorem{kor}[bsp]{Korollar}
\newtheorem{prob}[bsp]{Problem}

\numberwithin{equation}{section}
\numberwithin{algorithm}{section}
\numberwithin{figure}{section}
\numberwithin{table}{section}


%\addto\captionsngerman{\renewcommand{\refname}{Literatur}}
%\addto\captionsngerman{\renewcommand*\contentsname{Inhalt}}
%\addto\captionsngerman{\renewcommand{\listfigurename}{\large Abbildungsverzeichnis}}
%\addto\captionsngerman{\renewcommand{\listtablename}{\large Tabellenverzeichnis}}
%\renewcommand{\listalgorithmname}{\large Algorithmenverzeichnis}

%\DefineBibliographyStrings{german}{andothers={et al.}}
%\urlstyle{same}


%\usepackage{ucs}
%\usepackage{multibib}
\usetikzlibrary{shapes,arrows}
\usetikzlibrary{decorations}


