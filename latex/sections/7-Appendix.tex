
\appendix
\section{Overview XAI Approaches}\label{sec:7}

\subsection{Inherently interpretable machine learning models}

\subsubsection{Linear Regression}
\begin{table}[H]
  \centering
  \begin{tabular}{|p{0.17\linewidth}|p{0.77\linewidth}|}
    \hline
    Form of \newline explanation & 
    Internal model parameters \\
    
    \hline
    Interpretation & 
    An increase or change of feature  $x_k$  by one unit increases the prediction for $y$ by $\beta_k$  units when all other feature values remain fixed. \\
    \hline
    Advantages &
    \begin{itemize}[nosep, left=0em]
        \item transparent 
        \item solid statistical history
        \item extensions: e.g. GLM, GAM (less interpretable)
    \end{itemize} \\
    
    \hline
    Disadvantages &
    \begin{itemize}[nosep, left=0em]
        \item can only represent linear relationships 
        \item not suitable if features correlations to strong
        \item interactions must be added manually
    \end{itemize} \\
    
    \hline
    Properties & 
    intrinsic, model-specific, global  \\
    
    \hline
  \end{tabular}
  \caption{Overview XAI - Linear Regression.}
  % \label{tab:XAILinReg}
\end{table}

\subsubsection{Decision Tree}
\begin{table}[H]
  \centering
  \begin{tabular}{|p{0.17\textwidth}|p{0.77\textwidth}|}
    \hline
    Form of \newline explanation & 
    Internal model parameters \\
    
    \hline
    Interpretation & 
    If feature $x$ is [smaller/bigger] than threshold $c$ AND $\dots$ then the predicted outcome is the mean value of y of the instances in that node. \\
 
    \hline
    Advantages &
    \begin{itemize}[nosep, left=0em]
        \item captures interactions between features in the data
        \item interpretation easy, ends up in distinct groups, good explanations
    \end{itemize} \\
    
    \hline
    Disadvantages &
    \begin{itemize}[nosep, left=0em]
        \item fails to deal with linear relationships
        \item lack of smoothness
        \item unstable, few changes in the training dataset can create a completely different tree
        \item number of terminal nodes increases quickly with depth, which makes trees more difficult to understand
    \end{itemize} \\
    
    \hline
    Properties & 
    intrinsic, model-specific, global  \\
    
    \hline
  \end{tabular}
  \caption{Overview XAI - Decision Tree.}
  \label{tab:Tree}
\end{table}

\subsubsection{Linear Tree}
\begin{table}[H]
  \centering
  \begin{tabular}{|p{0.17\textwidth}|p{0.77\textwidth}|}
    \hline
    Form of \newline explanation & 
    Internal model parameters \\
    
    \hline
    Interpretation & 
    Combination of Decision Tree and Linear Regression. If feature $x$ is [smaller/bigger] than threshold $c$ AND $\dots$ then the predicted outcome is the prediction of a Linear Regression model where the weights are interpretable.\\
 
    \hline
    Advantages &
    \begin{itemize}[nosep, left=0em]
        \item interpretation easy
        \item get insights into linear and non-linear relationships
        \item identify subpopulations with different behavior
    \end{itemize} \\
    
    \hline
    Disadvantages &
    \begin{itemize}[nosep, left=0em]
        \item as the number of terminal nodes gets higher,  makes Linear Tree more difficult to understand
    \end{itemize} \\
    
    \hline
    Properties & 
    intrinsic, model-specific, global  \\
    
    \hline
  \end{tabular}
  \caption[Overview XAI - Linear Tree]{Overview XAI - Linear Tree.\cite{linear_tree}}
\end{table}

\subsubsection{ProtoPNet - \textit{This} Looks Like \textit{That}}
\begin{table}[H]
  \centering
  \begin{tabular}{|p{0.17\textwidth}|p{0.77\textwidth}|}
    \hline
    Form of \newline explanation & Points from dataset distribution.
    \\
    
    \hline
    Interpretation & Learned prototypical parts are compared to points from training dataset.
     \\
 
    \hline
    Advantages &
    \begin{itemize}[nosep, left=0em]
        \item can achieve comparable accuracy with its analogous non-interpretable counterpart
        \item provides a level of interpretability that is absent in other interpretable deep models    
    \end{itemize} \\
    
    \hline
    Disadvantages &
    \begin{itemize}[nosep, left=0em]
        \item (minimal) trade-off in accuracy to best-performing deep models
    \end{itemize} \\
    
    \hline
    Properties & intrinsic, model-specific, local
      \\
    
    \hline
  \end{tabular}
  \caption[Overview XAI - ProtoPNet]{Overview XAI - ProtoPNet.\cite{NEURIPS2019_adf7ee2d}}
  % \label{tab:XAIProtoPNet}
\end{table}

\subsection{Post-hoc methods - Model-Agnostic - global}

\subsubsection{PFI - Permutation Feature Importance}
\begin{table}[H]
  \centering
  \begin{tabular}{|p{0.17\textwidth}|p{0.77\textwidth}|}
    \hline
    Form of \newline explanation & 
    Feature statistics \\
    
    \hline
    Interpretation & 
    A feature is important if shuffling its values increases the model error. \\
    \hline
    Advantages &
    \begin{itemize}[nosep, left=0em]
        \item provides a highly compressed, global insight into the model’s behavior
        \item takes into account all interactions with other features
    \end{itemize} \\
    
    \hline
    Disadvantages &
    \begin{itemize}[nosep, left=0em]
        \item when the permutation is repeated, the results might vary greatly
        \item can be biased by unrealistic data instances
        \item adding a correlated feature can decrease the importance of the associated feature by splitting the importance between both features
    \end{itemize} \\
    
    \hline
    Properties & 
    post-hoc, model-agnostic, global  \\
    
    \hline
  \end{tabular}
  \caption{Overview XAI - Permutation Feature Importance.}
  % \label{tab:XAIPFI}
\end{table}

\subsubsection{PDP - Partial Dependence Plot}
\begin{table}[H]
  \centering
  \begin{tabular}{|p{0.17\textwidth}|p{0.77\textwidth}|}
    \hline
    Form of \newline explanation & 
    Features statistics \\
    
    \hline
    Interpretation & 
    Feature effect method.\\
    \hline
    Advantages &
    \begin{itemize}[nosep, left=0em]
        \item clear interpretation
    \end{itemize} \\
    
    \hline
    Disadvantages &
    \begin{itemize}[nosep, left=0em]
        \item does not show the feature distribution
        \item it is assumed that the feature(s) for which the partial dependence is computed are not correlated with other features
        \item heterogeneous effects might be hidden
    \end{itemize} \\
    
    \hline
    Properties & 
    post-hoc, model-agnostic, global  \\
    
    \hline
  \end{tabular}
  \caption{Overview XAI - Partial Dependence Plot (PDP).}
  \label{tab:XAIPDP}
\end{table}

\subsubsection{Global surrogate}
\begin{table}[H]
  \centering
  \begin{tabular}{|p{0.17\textwidth}|p{0.77\textwidth}|}
    \hline
    Form of \newline explanation & 
    Approximation with an inherently interpretable model \\
    
    \hline
    Interpretation & 
    The original model is replaced with a simpler model for interpretation.\\
    \hline
    Advantages &
    \begin{itemize}[nosep, left=0em]
        \item any interpretable model can be used
        \item with the R-squared measure, we can easily measure how good our surrogate models are in approximating the black box predictions
    \end{itemize} \\
    
    \hline
    Disadvantages &
    \begin{itemize}[nosep, left=0em]
        \item you conclude the model and not about the data
        \item what is the best cut-off for R-squared?
        \item could happen that the interpretable model is very close for one subset of the dataset but widely divergent for another subset
    \end{itemize} \\
    
    \hline
    Properties & 
    post-hoc, model-agnostic, global  \\
    
    \hline
  \end{tabular}
  \caption{Overview XAI - Global surrogate.}
  \label{tab:XAIGlobSur}
\end{table}

\subsection{Post-hoc methods - Model-Agnostic - local}

\subsubsection{Shapley Values}
\begin{table}[H]
  \centering
  \begin{tabular}{|p{0.17\textwidth}|p{0.77\textwidth}|}
    \hline
    Form of \newline explanation & 
    Feature statistics. \\
    
    \hline
    Interpretation & 
    Given the current set of feature values, the contribution of a feature value to the difference between the actual prediction and the mean prediction is the estimated Shapley value. \\
    \hline
    Advantages &
    \begin{itemize}[nosep, left=0em]
        \item prediction is fairly distributed among the feature values
        \item has a solid theoretical foundation in game theory
        \item can be used for global model interpretations
    \end{itemize} \\
    
    \hline
    Disadvantages &
    \begin{itemize}[nosep, left=0em]
        \item requires a lot of computing time
        \item no sparse explanation, always use all the features
        \item problem with correlated features, can include unrealistic data instances
    \end{itemize} \\
    
    \hline
    Properties & 
    post-hoc, model-agnostic, local  \\
    
    \hline
  \end{tabular}
  \caption{Overview XAI - Shapley Values.}
  % \label{tab:XAIShapVal}
\end{table}

\subsubsection{ICE - Individual Conditional Expectation}
\begin{table}[H]
  \centering
  \begin{tabular}{|p{0.17\textwidth}|p{0.77\textwidth}|}
    \hline
    Form of \newline explanation & 
    Feature visualization. \\
    
    \hline
    Interpretation & 
    Shows how the prediction of one instance changes when a feature changes.\\
    \hline
    Advantages &
    \begin{itemize}[nosep, left=0em]
        \item intuitive to understand
        \item can uncover heterogeneous relationships
    \end{itemize} \\
    
    \hline
    Disadvantages &
    \begin{itemize}[nosep, left=0em]
        \item problem with correlated features, some points in the lines might be invalid data points
        \item when many ICE curves are plotted, the visual representation may become chaotic
    \end{itemize} \\
    
    \hline
    Properties & 
    post-hoc, model-agnostic, local  \\
    
    \hline
  \end{tabular}
  \caption{Overview XAI - Individual Conditional Expectation (ICE)}
  \label{tab:XAIICE}
\end{table}

\subsubsection{LIME - Local Surrogate.}
\begin{table}[H]
  \centering
  \begin{tabular}{|p{0.17\textwidth}|p{0.77\textwidth}|}
    \hline
    Form of \newline explanation & 
    Feature visualization. \\
    
    \hline
    Interpretation & 
    Train a local surrogate model to explain individual prediction.\\
    \hline
    Advantages &
    \begin{itemize}[nosep, left=0em]
        \item you can use interpretable models to provide explanations without necessarily using them for prediction purposes
        \item local surrogate models can use features beyond those used in the original model during training.
    \end{itemize} \\
    
    \hline
    Disadvantages &
    \begin{itemize}[nosep, left=0em]
        \item for each application you have to try different kernel settings and see for yourself if the explanations make sense
        \item data points are sampled from a Gaussian distribution, ignoring the correlation between features, which can lead to unlikely data points
        \item by repeating the sampling process, the resulting explanations can vary
        \item LIME explanations can be manipulated to hide biases\cite{slack2020fooling}
    \end{itemize} \\
    
    \hline
    Properties & 
    post-hoc, model-agnostic, local  \\
    
    \hline
  \end{tabular}
  \caption{Overview XAI - Individual Local Surrogate (LIME).}
  \label{tab:XAILIME}
\end{table}







\subsection{Post-hoc methods - Model-Specific}


\subsubsection{Image-Specific Class Saliency (Vanilla gradient)}

\begin{table}[H]
  \centering
  \begin{tabular}{|p{0.17\textwidth}|p{0.77\textwidth}|}
    \hline
    Form of \newline explanation & 
    Feature visualization \\
    
    \hline
    Interpretation & 
    Pixels that strongly influence the prediction of the neural network are identified. \\
    \hline
    Advantages &
    \begin{itemize}[nosep, left=0em]
        \item easy to immediately recognize the important regions of the image to the model's prediction
    \end{itemize} \\
    
    \hline
    Disadvantages &
    \begin{itemize}[nosep, left=0em]
        \item tells only where the network is looking not why the prediction was made
        \item can be highly unreliable
    \end{itemize} \\
    
    \hline
    Properties & 
    post-hoc, model-specific, local  \\
    
    \hline
  \end{tabular}
  \caption[Overview XAI - Image-Specific Class Saliency (Vanilla gradient)]{Overview XAI - Image-Specific Class Saliency (Vanilla gradient).}
\end{table}
